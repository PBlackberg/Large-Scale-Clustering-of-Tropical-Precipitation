\documentclass[12pt]{article}
\begin{document}
\section*{Response to Reviewers}
\subsection*{Reviewer 1}
Thank you for the insightful comments.
Here is a list of revisions that have been implemented in response.

\subsubsection*{General comments}
\textbf{1a. True aggregation effects: dimensions and combined measures}
As the reviewer points out, there is some ambiguity in what high values of Am implies about the spatial distribution of heavy rainfall in interannual variability.
In interannual variability, Am is a combination of the area covered by heavy rainfall and increased proximity of rainy points sufficient to connect contiguous precipitation features:
\[ Am = Am_{coverage} + Am_{proximity} \]
These two "dimensions" of clustering are evaluated in the paper using \\
$ Am_{coverage} \propto Af $ \\
$ Am_{proximity} \propto Am|Af $ \\
\\
The reason for decomposing clustering into these two "dimensions" is that they are largely driven by different mechanisms.
Af is to first order described by the tropical mean rainfall rate and can be viewed as a temporal clustering of heavy rainfall, locally increasing the size of- without spatial clustering precipitation features, indicated by strong linear relationship between Am and Af (Figure 3a):\\
\[ Am_{coverage} = \alpha Af \]
$ Am_{coverage} = (\frac{1}{N^*})Af $ \\
Where \\
Af - Area covered by heavy precipitation \\
N* - Characteristic number of precipitation features (constant) \\ 
\\
The deviation from this local growing tendency, that is, the tendency in Am that cannot be explained by a linear response from changes in Af, represents the increase in proximity of heavily raining points, which we refer to as spatial clustering:\\
\[ Am_{proximity} = Am - \alpha Af \]

There are varying interpretations of clustering, some dominated by one dimension and some a combination of dimensions with different signs.
Here we explore one of these combined measures (Am) and its two orthogonal components (Af, Am$|$Af).
Am and N are both examples of a combined measure and they are directly dependent on each other as reciprocals:
\[ Am = (\frac{Af}{N}) \]
\[ Am = \propto (\frac{1}{N}) \]
Therefore, we only focus on one of these measures, and choose Am as it was the centre of the preceding study on large-scale clustering with warming.\\

\emph{Edits:}
\begin{enumerate}
\item Added a schematic illustrating these two dimensions of clustering in Figure 3
\item Added explanation in lines; 
\item Further, we have condensed the clustering measures included in the Supporting information Figure S1, S2, to the ones central to the paper.
\end{enumerate}


\clearpage
\textbf{1b. Justification and sensitivity testing of precipitation threshold}
The 5\% precipitation threshold is somewhat arbitrarily picked. The purpose of the threshold is to isolate rainfall associated with considerable convection, to relate to previous work on organization of convection and its associated environmental signature.
Early in this work we have explored 3\%, 5\% and 10\% thresholds, and during this revision all figures from the paper have been reproduced with these thresholds. The overall conclusions from the paper are not sensitivie to the threshold, including; 
uniform increase in Am with warming, 
increased proximity to the equator and hydrological equator, 
El Nino conditions associated with increased clusterng in interannual variability and with warming, 
and discrepencies between modelled and observed relative humidity and cloud signatures associated with clustering in interannual variability.  
Differences include, when using the top 10\%, the drying associated with Af in interannual variability is weaker and not statistically significant, instead Am|Af dries the tropics when using this threshold.
Further, the Cz relationship with Am is absent when using the top 10\%. When using the top 3\% the proximity to the hydrological equator (Cheq) and Cz explain close to 20 of the variance each in Am with warming, and Cheq is anticorrelated with changes in Am. This was suspected, due to
Cz and Cheq being somewhat anticorrelated.  

\vspace{1cm}
\textbf{2. Title edit}
The title has been changed to better reflect the study scope.
\emph{Edits:}
line: 


\subsubsection*{Specific comments}






















% Comments:
% A commmon cause for disagreement between different measures is the interpretation of reducing Af as clustering. 
% In our view that interpretation only makes sense when the same amount of rainfall is concentrated into a small region, not when comparing vastly differnt amount of rainfall.

















































\end{document}









