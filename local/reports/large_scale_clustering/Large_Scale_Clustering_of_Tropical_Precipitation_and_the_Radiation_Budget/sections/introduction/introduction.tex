\section{Introduction}
% Motivation for looking at clustering
The spatial organization of deep convection in tropical regions plays a critical role in shaping the hydrological cycle and the moisture and cloud distribution \citep{Hartmann1984}. 
Changes in organization with warming may therefore have implications for a range of climatic processes, including precipitation extremes \citep[e.g.,][]{Pendergrass2016, Bao2017, Semie2020} 
and the radiative feedbacks that control equilibrium climate sensitivity (ECS) \citep[e.g.,][]{Emanuel2014, Bony2020, Schiro2022}. 
However, because many of the relevant small-scale processes are not resolved in climate models, it remains unclear how convective organization will evolve in a warmer climate. 

% Large-scale clustering: importance and limitations
While there are numerous ways by which convection may organize, one important mechanism is the clumping or clustering together of convective elements \citep[e.g.,][]{Maddox1980, Mapes1993, Bretherton2005}. 
Such clustering occurs on a range of scales \citep{Mapes1993_cloud}, including at large scales that are resolved by climate models and at mesoscales that can typically only be resolved in high-resolution storm-resolving simulations. 
Recently, \citet{Blackberg_Singh2022} showed that the extent to which tropical precipitation exhibits clustering on the large scale increases with warming in climate projections from the Coupled Model Intercomparison Project phase 5 (CMIP5). 
This large-scale clustering is distinct from other types of organization on the mesoscale, but idealized simulations suggest that similar processes may act at both scales, and that both large-scale and mesoscale organization of convection may modulate the radiation budget \citep{Wing2018}. 

% Short paper summary
Here we build on the work of \citet{Blackberg_Singh2022}, showing that increased clustering of heavy precipitation with warming is a robust feature of the more recent CMIP6 as well as CMIP5. 
Further, we explore the mechanisms that lead to large-scale clustering of precipitation in the tropics and the influence of an increase in clustering on properties of the atmosphere that are important for the radiation budget. 
The analysis will compare how clustering varies across different timescales, from interannual variability in both models and observations to changes in the climatological clustering of convection in a warming climate.
This approach allows us to assess whether observational constraints of convective organization under current climate conditions can help constrain changes in organization and the associated radiative feedbacks with warming.

% Radiative feedbacks in interannual variability: Observations
Previous studies have highlighted observed relationships between convective organization and the tropical radiation budget \citep{Tobin2013, Holloway2017, Bony2020}.
For example, \citet{Bony2020} find tropical mesoscale convective organization and Estimated Inversion Strength (EIS) in subsidence regions are the two strongest predictors of 
deseasonalized \change[\firstAuthorEdit]{monthly anomalies}{interannual variability} in net top-of-atmosphere radiation, together explaining about 60 percent of the variance. 
While the two predictors are significantly correlated and potentially partly mechanistically connected \citep{Williams2023}, the authors find that both have an independent contribution in influencing the tropical radiation budget; 
EIS is found to have a stronger correlation with the cloudy component of the radiation budget while convective organization is found to have a stronger connection to the clear-sky component of the radiation budget. 
\citet{Bony2020} argue that clustering of deep convective elements is associated with a tropical-mean drying, resulting in increased outgoing longwave radiation due to a reduction in the greenhouse effect. 

% Radiative feedbacks in interannual variability and climatology: Models
This hypothesis is supported by idealized studies of convective ``self-aggregation'' \citep[e.g.,][]{wing2014}. 
Both cloud-resolving and climate-model simulations run in idealized settings reminiscent of tropical conditions (i.e., low rotation rate and weak temperature gradients) 
show increased outgoing longwave radiation when convection is more clustered within the domain \citep{Wing2018}.

% Climate sensitivity: through changes in relative humidity and clouds
The preceding studies suggest that increased clustering of convection with warming may lead to a negative clear-sky feedback from clustering-induced drying, resulting in reduced equilibrium climate sensitivity (ECS) \citep{Emanuel2014}. 
However, recent research suggests that clustering of deep convection on the large scale may also be indirectly connected to a positive shortwave feedback on warming through changes in low clouds \citep{Schiro2022}. 
According to this argument, drying associated with increased clustering of convection is controlled by the large-scale overturning circulation and is most pronounced in regions of climatological descent where low-cloudiness is sensitive to changes in relative humidity.
The associated cloud changes then lead to a net positive feedback.

% Mechanisms: including Pacific SST and a narrowing of the ITCZ
As the above discussion highlights, an important control on convective organization at both large- and mesoscales comes from large-scale circulation patterns, 
including the Intertropical Convergence Zone (ITCZ), 
the South Pacific Convergence Zone (SPCZ), 
the Walker circulation, and 
convectively-coupled tropical waves \citep{Bony2020, Quan2025, Arnold2015, Wodzicki2016, Wheeler1999}. 
Changes to the clustering of precipitation under warming may therefore be linked to, for example, a ``narrowing'' of the ITCZ due to constraints on the export of energy by the Hadley cell \citep{Byrne2016} or 
changes in the Walker circulation driven by changes in zonal SST gradients \citep{Quan2025}, 
which may be associated with ``El Ni\~no-like'' shifts in the SST climatology \citep{Watanabe2024}. 

\change[\firstAuthorEdit]{
In this study we explore how the spatial distribution of heavy precipitation changes with increased large-scale clustering, demonstrating the importance of both meridional contractions and zonal shifts in convection. 
We further show that the effect of large-scale clustering on low clouds and moisture---key variables that control the radiation budget---
varies with timescale and between global climate models and observations, 
making it difficult to apply constraints from observed variability to a warmer climate. 
The paper is structured as follows. 
First we describe the datasets used and our quantification of large-scale clustering of precipitation (Section 2). 
Then we present the geographical spatial patterns of heavy precipitation that are associated with a high level of clustering (Section 3). 
After that, we connect the spatial patterns to leading mechanisms driving clustering (Section 4). 
Finally, we present the moisture and low-cloud distribution associated with clustering (Section 5).  
Section 6 gives a summary of the key findings and an outlook for future research.
}{
% Researh gap
The existing literature show large-scale clustering is projected to increase from the forced response of global warming, and that a more clustered state in observed interannual variability, albeit on a different spatial scale, 
is associated with clear-sky and cloud-radiative feedbacks connected to climate sensitivity. 
% Question
In this study we elucidate to what extent observed and modelled interannual variability in large-scale clustering of heavy precipitation connects to similar radiative feedbacks and can be used to constrain climate sensitivity.
% Novelty and structure of paper
To do so, we quantify and relate clustering of heavy rainfall to the distribution and tropical-mean relative humidity and low-cloudiness in observations, a high-resolution GCM from the NextGEMS pre-final cycle, and GCMs from the CMIP6 ensemble (method described in Section 2).
The results of this study adds to the existing literature by describing 
the spatial preference of heavy rainfall that promote large-scale clustering (Section 3), 
the associated mechanisms involving SST patterns and a narrowing of the ITCZ (Section 4), 
and the tropical radiative feedbacks associated with a more clustered state (Section 5) 
in interannual variability and for projected climatological changes with warming.
Section 6 gives a summary of the key findings and an outlook for future research.
}




% == Track changes commands ==
%  \note[editor]{The note}
%  \annote[editor]{Text to annotate}{The note}
%  \add[editor]{Text to add}
%  \remove[editor]{Text to remove}
%  \change[editor]{Text to remove}{Text to add}
% complete documentation is here: http://trackchanges.sourceforge.net/

% ex:
% \change[\firstAuthorEdit]{Text to remove}{Text to add}



% == Extra ==


