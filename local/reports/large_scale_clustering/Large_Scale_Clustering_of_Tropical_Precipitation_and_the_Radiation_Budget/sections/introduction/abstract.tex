\begin{abstract}
% Motivation
The spatial organization of deep convection in tropical regions is posited to play an important role in determining 
characteristics of the tropical climate such as the humidity distribution and cloudiness and 
may therefore be an important control on climate feedbacks. 
% Main methods
This study analyzes one aspect of convective organization, the clustering of heavy precipitation on large scales, 
in both interannual variability and under warming in future climate projections. 
\add[\firstAuthorEdit]{
Clustering is quantified using the top 5\% heaviest daily precipitation instances, 
and interpreted as increasing with the area covered by heavy precipitation ($C$, temporally clustered)
and proximity of heavily rainy points to each other and key zonal and meridional reference lines ($P$, spatially clustered and orthogonal to coverage).
}
\change[\firstAuthorEdit]{Both observations and global climate models indicate that large-scale clustering is sensitive to the SST gradient in the Pacific, being largest during El Ni\~no events. 
Under future warming, models project an increase in clustering with a large intermodel spread. 
The increase is associated with a narrowing of the intertropical convergence zone, 
while the model spread is partially explained by differences in projections of the SST gradient in the Pacific. 
Both observations and models indicate large-scale clustering influences the cloud and humidity distributions, 
albeit with some differences. 
However, the intermodel spread in changes in clustering with warming is not related to the intermodel spread in projections of tropical-mean relative humidity or low cloudiness in regions of descent, 
precluding attempts to provide an observational constraint on feedbacks or climate sensitivity. 
Nevertheless, the tendency for a meridional contraction of precipitation explains about 45\% of the variance in projected drying, 
highlighting the narrowing of the ITCZ as an important aspect of large-scale convective organization in a warmer climate.  
}{
% Key findings
% (interannual variability)
Both observations and global climate models (GCMs) indicate temporal clustering of heavy rainfall is best described by the tropical-mean precipitation rate ($r \sim 0.55 - 0.85 $) and spatial clustering is favored by El Ni\~no conditions,
but importantly diverge in the tropical-mean relative humidity and low cloudiness associated with spatially clustered states in deseasonalized interannual variability.
% (changes with warming)
Under future warming, all models from CMIP6 ensemble project an increase in clustering associated with a narrowing of the intertropical convergence zone
and with a model-spread partially explained by differences in projections of the Pacific SST gradient (r=-0.46). 
However, unlike variability, changes in general spatial clustering with warming do not explain the climatological humidity or low-cloud responses, 
% Implications
limiting a simple observational constraint on feedbacks.
Notably a few model outliers drive considerable spread in tropical-mean drying (r=0.44) linked to a climatological meridional clustering of heavy rainfall, suggesting potential for future model evaluation.
}
\end{abstract}




% == Track changes commands ==
%  \note[editor]{The note}
%  \annote[editor]{Text to annotate}{The note}
%  \add[editor]{Text to add}
%  \remove[editor]{Text to remove}
%  \change[editor]{Text to remove}{Text to add}
% complete documentation is here: http://trackchanges.sourceforge.net/

% ex:
% \change[\firstAuthorEdit]{Text to remove}{Text to add}





% == Extra ==
% % Main methods
% Clustering is quantified using the top 5\% heaviest daily precipitation instances, 
% and interpreted as increasing when an increased total area is covered (temporally clustered) and when
% heavily precipitating points move closer to each other or key meridional and zonal reference lines (spatially clustered, orthogonal to coverage).

% % Key findings (interannual variability)
% In both observed and modelled deseasonalized interannual variability,
% the area covered by heavy precipitation is to first order described by the tropical-mean precipitation rate (r$\sim$0.85), and spatial clustering is generally favoured by El Nino conditions (r $\sim$ 0.2 $-$ 0.25).
% However, observations and models importantly disagree on the associated tropical mean relative humidity and low cloudiness of highly spatially clustered states, with a mix of signs within the model ensemble.

% % Key findings (changes withh warming)
% Climatologically the CMIP ensemble project a uniform increase in spatial clustering with warming and exhibit a wide spread in tendencies.
% Our results suggest the uniform increase in spatial clustering is associated with a narrowing of the ITCZ, through a uniform meridional clustering of heavy precipitation and reduction in the area of ascent (r $\sim$ 0.6),
% while the somewhat anti-correlated zonal clustering best describe the model spread through a weakening of the Pacific SST gradient (r $\sim$ -0.45).

% % Implications
% Unlike interannual variability, a general spatial clustering with warming cannot explain changes in low cloudiness and relative humidity in a climatological sense, 
% precluding a simple observational constraint on feedbacks or climate sensitivity.
% However, we find a few model outliers create considerable spread in tropical-mean relative drying (r $\sim$ 0.44) with projected meridional clustering, 
% suggesting these convective spatial preference tendencies may be used for model evaluation in future work.

% in observations, the state-of-the-art CMIP ensemble, and one high-resolution GCM. 

% The tropical area covered by heavy precipitation is connected to a tropically drier domain-mean and increased subsidence low cloud fraction (r $\sim$ $\pm$ 0.25), 
% while spatial clustering is solely weakly connected to an increase in low cloud fraction in observed interannual variability (r $\sim$ 0.15).   


% In observed detrended and deseasonalized interannual variability, 
% heavy precipitation occupy a greater area to first order described by the tropical-mean precipitation rate (r$\sim$0.85), and 
% general spatial clustering is promoted by both greater meridional proxmimity to the equator and zonal proximity to the central pacific, associated with strong El Nino conditions (r $\sim$ 0.2 $-$ 0.25).
% Modelled and observed interannual variability generally agree El Nino conditions favour a more clustered state and connect clouds, humidity, and high area coverage of heavy precipitation in the same way, 
% but importantly disagree in the way spatial clustering influence tropical mean humidity and low cloudiness with a mix of signs within the model ensemble.










