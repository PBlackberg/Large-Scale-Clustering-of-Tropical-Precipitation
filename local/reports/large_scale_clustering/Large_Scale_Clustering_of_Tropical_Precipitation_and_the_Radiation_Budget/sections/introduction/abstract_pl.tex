\section*{Plain Language Summary}
The spatial distribution of rainfall in the tropics is expected to change in a warming climate, 
with potentially important impacts on how much radiation is absorbed by water vapor and reflected by clouds. 
This study shows that heavy rainfall tends to move towards the equator and to the Pacific Ocean in projections with global climate models, 
resulting in an overall increase in the "clustering" of rainfall on the large scale. 
Further, the results show a shift in rainfall to the equator with global warming is associated with a drying of the tropical atmosphere, 
which may have an influence on how much the planet warms for a given $CO_2$ change. 
However, similar observed shifts in rainfall in the current climate are not found to have the same effect on humidity and clouds as for changes with warming, 
suggesting caution should be exercised when using relationships derived from observations to predict future changes.











