
\begin{figure}
    \includegraphics[]{sections/result_3/supplementary_figures/corr_matrix_interannual_var.pdf}  % [width=\linewidth]
    \caption{Correlation matrix for interannual variability, in 
    observations (green), 
    IFS\_9\_FESOM\_5 (purple), 
    and the CMIP ensemble mean and standard deviation (black). 
    The colors represent the sign and strength of the correlations from the observations.
    Numbers in brackets are not statistically significant from zero, and the fraction in brackets show the number of models with statistically significant correlations.
    Details of the metrics are given in Table S2. 
    }
\end{figure}

\begin{figure}
    \includegraphics[]{sections/result_3/supplementary_figures/corr_matrix_change.pdf}  % [width=\linewidth]
    \caption{
    Correlation matrix for climatological change with warming across the CMIP6 ensemble.
    All metrics are normalized by the change in tropical-mean temperature (land and ocean) except for tropical-mean temperature itself and ECS.
    Numbers in brackets are not statistically different from zero.    
    }
\end{figure}

\begin{figure}
    \includegraphics[]{sections/result_1/supplementary_figures/spatial_preference_month_breakdown.pdf}  % [width=\linewidth]
    \caption{GPCP Frequency of occurrence of heavy precipitation, $FOO$, regressed onto the mean area of heavy precipitation features, $A_m$, for each month. 
    Crosses indicate whether correlations are statistically significant.}
\end{figure}

\begin{figure}
    \includegraphics[]{sections/result_1/supplementary_figures/spatial_preference_interannual_cmip_ifs.pdf}  % [width=\linewidth]
    \caption{Frequency of occurrence of heavy precipitation, $FOO$, regressed onto the mean area of heavy precipitation features, $A_m$, 
    in interannual variability for the CMIP ensemble-mean, (a) and IFS\_9\_FESOM\_5 (b). 
    The contour shows the 90th percentile of the climatological $FOO$ and crosses indicate whether correlations are statistically significant.}
\end{figure}

\begin{figure}
    \includegraphics[]{sections/result_1/supplementary_figures/spatial_preference_cmip_clim.pdf}  % [width=\linewidth]
    \caption{Frequency of occurrence of heavy precipitation, $FOO$, regressed onto the mean area of heavy precipitation features, $A_m$, 
    across the CMIP ensemble in climatological values. The contour shows the ensemble-mean 90th percentile of the climatological $FOO$ and 
    crosses indicate whether correlations are statistically significant.}
\end{figure}

\begin{figure}
    \includegraphics[]{sections/result_1/supplementary_figures/nino_diff_maps.pdf}  % [width=\linewidth]
    \caption{Difference in frequency of occurrence of heavy precipitation, $FOO$, during El Ni\~no events and during all days 
    in the CMIP6 ensemble-mean, 
    in observations: GPCP, 
    and for the high-resolution model: IFS\_9\_FESOM\_5. 
    }
\end{figure}

\begin{figure}
    \includegraphics[]{sections/result_3/supplementary_figures/cloud_radiative_effect.pdf}  % [width=\linewidth]
    \caption{CERES-SYN1deg top of the atmosphere net radiative fluxes for all (a), cloud-radiative (b), and clear-sky (c) conditions regressed onto mean area of heavy precipitation features, $A_m$.
    Crosses indicate whether correlations are statistically significant.}
\end{figure}

\begin{figure}
    \includegraphics[]{sections/result_3/supplementary_figures/am_cm_cz_cheq_lcf_rh_map.pdf}  % [width=\linewidth]
    \caption{Change in relative humidity (RH, left) and low cloud fraction (LCF, right) regressed onto; $A_m$ (a-b), $P_z$ (c-d), $P_{eq}$ (e-f), and $P_{heq}$ (g-h) per Kelvin warming across the CMIP ensemble.
    }
\end{figure}







