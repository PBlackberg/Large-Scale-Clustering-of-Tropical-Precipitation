

\begin{figure}
    \includegraphics[]{sections/result_3/supplementary_figures/corr_matrix_interannual_var.pdf}  % [width=\linewidth]
    \caption{
    The ij'th box show the correlation between metric in row i and column j for interannual variability in the CMIP ensemble. In each box, the top value shows ensemble-mean, the value below show standard deviation of correlations, and the fraction at the bottom shows what fraction of models have statistically significant correlations. Details of the metrics are given in Table S2. 
    }
\end{figure}

\begin{figure}
    \includegraphics[]{sections/result_3/supplementary_figures/corr_matrix_change.pdf}  % [width=\linewidth]
    \caption{
    The ij'th box show the correlation between metric in row i and column j for interannual variability in observations. Numbers in brackets are not statistically significant form zero.
    }
\end{figure}

\begin{figure}
    \includegraphics[]{sections/result_1/supplementary_figures/spatial_preference_month_breakdown.pdf}  % [width=\linewidth]
    \caption{GPCP Frequency of occurrence of heavy precipitation, $C$, regressed onto the mean area of heavy precipitation features, $A_m$, for each month. Crosses indicate whether correlations are statistically significant.}
\end{figure}

\begin{figure}
    \includegraphics[]{sections/result_1/supplementary_figures/spatial_preference_interannual_cmip_ifs.pdf}  % [width=\linewidth]
    \caption{Frequency of occurrence of heavy precipitation, $C$, regressed onto the mean area of heavy precipitation features, $A_m$, in interannual variability for the CMIP ensemble-mean (a) and IFS\_9\_FESOM\_5 (b), and across the CMIP ensemble in climatological values (c). The contour shows the 90th percentile of the climatological $C$ in (a-b) and the ensemble-mean 90th percentile of the climatological $C$ in (c). Crosses indicate whether correlations are statistically significant.}
\end{figure}

\begin{figure}
    \includegraphics[]{sections/result_1/supplementary_figures/spatial_preference_interannual_cmip_ifs.pdf}  % [width=\linewidth]
    \caption{Frequency of occurrence of heavy precipitation, $C$, regressed onto the mean area of heavy precipitation features, $A_m$, in interannual variability for the CMIP ensemble-mean (a) and IFS\_9\_FESOM\_5 (b), and across the CMIP ensemble in climatological values (c). The contour shows the 90th percentile of the climatological $C$ in (a-b) and the ensemble-mean 90th percentile of the climatological $C$ in (c). Crosses indicate whether correlations are statistically significant.}
\end{figure}

\begin{figure}
    \includegraphics[]{sections/result_2/supplementary_figures/nino_diff_maps.pdf}  % [width=\linewidth]
    \caption{
Difference in frequency of occurrence of heavy precipitation, $C$, during El Ni\~no events and during all days in the CMIP ensemble-mean, in observations, and for the high-resolution model: IFS\_9\_FESOM\_5. 
    }
\end{figure}

\begin{figure}
    \includegraphics[]{sections/result_3/supplementary_figures/am_cm_cz_cheq_lcf_rh_map.pdf}  % [width=\linewidth]
    \caption{
Change in relative humidity (RH, left) and low cloud fraction (LCF, right) regressed onto; $A_m$ (a-b), $C_z$ (c-d), and $C_m$ (e-f), per Kelvin warming across the CMIP ensemble.
    }
\end{figure}







