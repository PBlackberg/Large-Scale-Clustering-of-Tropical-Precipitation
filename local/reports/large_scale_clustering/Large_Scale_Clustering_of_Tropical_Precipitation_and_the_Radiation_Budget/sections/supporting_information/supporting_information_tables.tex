\begin{table*}
\settablenum{S1}
\caption{Models from the CMIP6 archive that were used in this study.}
\centering
\begin{tabular}{l l l}
\hline
\textbf {Institute} & \textbf{Model} & \textbf{Ensemble} \\
\hline
        INM & INM-CM5-0 & r1i1p1f1 \\
        CCCR-IITM & IITM-ESM & r1i1p1f1 \\
        CAS & FGOALS-g3 & r1i1p1f1 \\
        INM & INM-CM4-8 & r1i1p1f1 \\
        MIROC & MIROC6 & r1i1p1f1 \\
        MPI-M & MPI-ESM1-2-LR & r1i1p1f1 \\
        BCC & BCC-CSM2-MR & r1i1p1f1 \\
        NOAA-GFDL & GFDL-ESM4 & r1i1p1f1 \\
        MIROC & MIROC-ES2L & r1i1p1f2 \\
        NorESM2-LM & NCC & r1i1p1f1 \\
        MRI & MRI-ESM2-0 & r1i1p1f1 \\
        NOAA-GFDL & GFDL-CM4 & r1i1p1f1 \\
        CMCC & CMCC-CM2-SR5 & r1i1p1f1 \\
        CMCC & CMCC-ESM2 & r1i1p1f1 \\
        NUIST & NESM3 & r1i1p1f1 \\
        CSIRO & ACCESS-ESM1-5 & r1i1p1f1 \\
        CNRM-CERFACS & CNRM-ESM2-1 & r1i1p1f2 \\
        EC-Earth-Consortium & EC-Earth3 & r1i1p1f1 \\
        CNRM-CERFACS & CNRM-CM6-1 & r1i1p1f2 \\
        CNRM-CERFACS & CNRM-CM6-1-HR & r1i1p1f2 \\
        NIMS-KMA & KACE-1-0-G & r1i1p1f1 \\
        IPSL & IPSL-CM6A-LR & r1i1p1f1 \\
        CSIRO-ARCCSS & ACCESS-CM2 & r1i1p1f1 \\
        AS-RCEC & TaiESM1 & r1i1p1f1 \\
        NCAR & CESM2-WACCM & r1i1p1f1 \\
        CCCma & CanESM5 & r1i1p1f1 \\
        MOHC & UKESM1-0-LL & r1i1p1f2 \\
\hline
\end{tabular}
\label{tab:models}
\end{table*}


\begin{table*}
\settablenum{S2}
\caption{Metrics used in this study.}
\centering
\begin{tabular}{@{}l p{0.8\linewidth}@{}}
\hline
\textbf {Metric} & \textbf{description} \\
\hline
        $A_m$ & Mean area of heavy precipitation features. \\
        $A_f$ & Total area of heavy precipitation features. \\
        $A_m|C$ & Spatial clustering from changes in mean area, independent of total area coverage ($A_m|C = A_m - \alpha C$), where $\alpha$ is a constant. \\
        $P_z$ & Mean distance of heavy precipitation points to the meridian given by the longitude 180$^\circ$E. \\
        $P_{eq}$ & Mean distance of heavy precipitation points to the geographic equator. \\
        $P_{heq}$ & Mean distance of heavy precipitation points to the hydrological equator, where the hydrological equator is defined as the latitude of highest specific humidity at 700 hPa as a function of longitude for the associated month. \\
        $GINI$ & Relative dispersion measure of the precipitation field, quantifying the "uneveness" of the distribution normalized by the mean (GINI; Gini, 1912). \\
        $T_s$ & Tropical-mean 2m air temperature. \\
        $T_{500hpa}$ & Tropical-mean temperature at 500 hPa. \\
        $ONI$ & Three-month rolling average SST anomaly in the Ni\~no3.4 region (5$^\circ{}$S- 5$^\circ{}$N, 120$^\circ{}$-170$^\circ{}$W), relative to the full range of years used in the climatology. \\
        $SOI$ & standardized deseasonalized monthly anomalies of the surface pressure difference between Tahiti (17.6$^\circ{}$S, 149.6$^\circ{}$W) and Darwin (12.5$^\circ{}$S, 130.9$^\circ{}$E), calculated here as the three-month rolling average. \\
        $LCF_d$ & Tropical-mean low-cloud fraction, in regions where the monthly-mean vertical pressure velocity at 500 hPa is positive (in regions of descent). \\
        $RH_{500hpa}$ & Tropical-mean relative humidity at 500 hPa. \\
        $HCF_a$ & Tropical-mean high-cloud fraction, in regions where the monthly-mean vertical pressure velocity at 500 hPa is negative (in regions of ascent). \\
        $A_a$ & Area where the monthly-mean vertical pressure velocity at 500 hPa is negative, as a fraction of the tropical domain area (area of ascent). \\
        $ECS$ & Equilibrium Climate Sensitivity. \\
\hline
\end{tabular}
\label{tab:metrics}
\end{table*}


% $A_f$ & Total area of heavy precipitation features as a fraction of the tropical domain area. \\
% $N$ & Number of heavy precipitation features. \\
% $OLR$ & Tropical-mean outgoing longwave radiation. \\
% $F_{pr10}$ & Frequency of gridpoints exceeding 10 mm day$^{-1}$. \\
% $\sigma(A_f)$ & Standard deviation of total area fraction of heavy precipitation, $A_f$. \\
