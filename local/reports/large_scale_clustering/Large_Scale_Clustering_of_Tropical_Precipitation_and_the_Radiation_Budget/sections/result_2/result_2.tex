\section{SST Drivers of Heavy Rainfall Clustering}
This section investigates the extent to which changes in the El Ni\~no-Southern Oscillation can explain changes in clustering across timescales. 
We use the Oceanic Ni\~no Index \citep[ONI;][]{ONI_ref} to identify the state of the El Ni\~no-Southern Oscillation in interannual variability. 
The ONI represents the three-month rolling average SST anomaly in the Ni\~no3.4 region (5$^\circ{}$S- 5$^\circ{}$N, 120$^\circ{}$-170$^\circ{}$W), calculated here relative to the full range of years used in the climatology. 
ONI values greater than 0.5$^\circ{}$C represent El Ni\~no conditions and ONI values less than -0.5$^\circ{}$C represent La Ni\~na conditions. 
The climatological East-West Pacific SST gradient, defined as the time-mean difference between 
the SST in the western- (5$\degree$S - 5$\degree$N, 80$\degree$ - 150E) and eastern (5$\degree$S - 5$\degree$N, 180$\degree$ - 80$\degree$W) Pacific boxes, 
which we denote $T_z$, 
serves as an indicator for climatologically ``El Ni\~no-like'' conditions \citep{Watanabe2024}.
Observations show several indications that highly clustered states, corresponding to large values of the mean area of heavy precipitation features, $A_m$, are associated with El Ni\~no-like conditions. 
Firstly, SST regressed onto $A_m$ for interannual variability shows a pattern strongly reminiscent of an El Ni\~no SST signature \citep{Michael2002} (Figure \ref{mechanisms_obsMAP}). 
Secondly, during times of ONI exceeding 0.5$^\circ{}$C compared to all days, $A_m$ increases as heavy precipitation moves from the maritime continent towards the central Pacific (Figure S7b in supporting information). 
Finally, independent of changes in $A_f$, ONI shows a positive partial correlation with $A_m$ (Figure \ref{mechanisms_obsSCATTER}a). 
Most CMIP models capture the observed connection between ONI and $A_m$ (Figure S1 in the supporting information) and the independent contribution of ONI on $A_m$ outside the influence of $A_f$, 
as does the high-resolution GCM (Figure S3 in the supporting information and Figure \ref{mechanisms_obsSCATTER}b). 
Thus both models and observations show that large-scale clustering of precipitation is stronger during El Ni\~no than La Ni\~na. 
This is despite the fact that El Ni\~no represents a weakening of the Walker circulation and a weakening of tropical SST gradients, 
both of which are generally thought to facilitate the organization of convection on large scales.

\clearpage
\begin{figure}
    \centering
    \includegraphics{sections/result_2/mechanisms1.pdf}
    \caption{NOAA-GlobalTemp surface temperature, Ts, 
    regressed onto $A_m$ for interannual variability. 
    The contour shows the climatological 90th percentile of Ts, 
    and crosses indicate whether correlations are statistically significant.}
\label{mechanisms_obsMAP}
\end{figure}

\begin{figure}
    \centering
    \includegraphics{sections/result_2/mechanisms2.pdf}
    \caption{Same as Figure 3, 
    but with the Oceanic Ni\~no Index (ONI) in scatter colors 
    and as explanatory variable in boxplot.}
\label{mechanisms_obsSCATTER}
\end{figure}
\clearpage

The strong connection between El Ni\~no conditions and $A_m$ in interannual variability 
motivates the investigation of changes in the climatological Pacific SST gradient to a more El Ni\~no-like state as a mechanism for explaining model-spread in projected changes to clustering. 
Consistent with expectations, the magnitude of the weakening of the East-West Pacific SST gradient explains a similar amount of variance in projected changes in clustering as the zonal shift in heavy precipitation (Figure \ref{mechanisms_warmingSCATTER}). 
Models that have more El Ni\~no-like warming patterns tend to exhibit larger increases in large-scale clustering of precipitation. 
Regressing the SST changes against projected changes in the mean area of precipitation features, $A_m$, also shows an El-Ni\~no-like pattern, 
with a relative warming in the east and relative cooling in the west (Figure \ref{mechanisms_warmingMAP}). 
In addition, the regression pattern has a noticeable north-south gradient, 
consistent with the positive regression coefficients for heavy precipitation frequency north of the equator in Figure \ref{spatial_preference_warmingMAP}. 

We have shown that El Ni\~no-like states tend to result in a higher degree of clustering in both interannual variability and across the CMIP6 ensemble under climate change. 
Note, however, that there are changes in the zonal SST gradient, $T_z$, and the mean distance of heavy precipitation to the central Pacific of both signs across the 27 CMIP6 models we analyse, 
suggesting that zonal shifts in convection are not the primary reason for the ensemble-mean increase in large-scale clustering with warming that we document. 
We hypothesize that the ensemble-mean increase in $A_m$ is instead associated with a meridional shift in convection, 
potentially related to a narrowing of the ITCZ \citep{Byrne2016}. 
All but one model exhibit negative changes in $C_m$ with warming, and this is associated with an increase in the large-scale clustering of precipitation in natural variability.

\clearpage
\begin{figure}
    \centering
    \includegraphics{sections/result_2/mechanisms3.pdf}
    \caption{Same as Figure 4a, 
    but with the change in the climatological Pacific SST gradient, $T_z$, per Kelvin tropical warming from the historical to the SSP585 scenario simulation period across the CMIP6 ensemble as explanatory variable. 
    Models are as given in the legend in Figure 4.}
\label{mechanisms_warmingSCATTER}
\end{figure}

\begin{figure}
    \centering
    \includegraphics{sections/result_2/mechanisms4.pdf}
    \caption{Change in surface temperature, Ts, 
    regressed onto change in mean area of heavy precipitation features, $A_m$, per Kelvin tropical warming from the historical to the SSP585 scenario simulation period across the CMIP6 ensemble. 
    Contour shows ensemble-mean 90th percentile climatological Ts and crosses indicate if correlations are statistically significant.}
\label{mechanisms_warmingMAP}
\end{figure}
\clearpage



