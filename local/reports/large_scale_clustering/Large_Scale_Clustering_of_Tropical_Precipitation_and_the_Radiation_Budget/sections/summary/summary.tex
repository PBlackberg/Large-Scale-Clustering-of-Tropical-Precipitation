\section{Summary and Discussion}
In this paper we have 
(1) presented the dominant spatial patterns of heavy precipitation that produce a high degree of clustering on the large scale (Section 3); 
(2) tied the associated spatial patterns to mechanisms driving clustering through large-scale SST patterns (Section 4); and 
(3) evaluated the associated changes in properties of the atmosphere that are important for the radiation budget (Section 5) 
in both interannual variability and for projected changes with warming. 
We have defined the degree of clustering of precipitation based on the spatial distribution of the top 5 percent heaviest daily rainfall instances, with high clustering corresponding to scenes in which the mean area of individual precipitation features is large. 
A challenge in any definition of convective organization is in how one measures organization consistently as the total amount of precipitation changes \citep{Retsch2020, Tobin2013}. 
In the present study, the use of a percentile precipitation threshold accounts for changing mean precipitation rates in different climates. 
However, in internal variability, the area of precipitation features is affected by both spatial shifts in the precipitation distribution and variations in the total area fraction of heavy precipitation, $A_f$. 
This is addressed here by using Pearson partial correlations to evaluate the independent contributions of different measures of the spatial distribution of precipitation while controlling for the effect of $A_f$.

When tropical precipitation is observed to be highly clustered on the large scale in interannual variability, heavy precipitation gravitates meridionally to the equator and zonally towards the central Pacific. 
In climate projections, large-scale clustering of precipitation is found to increase in all models, and this coincides with a shift of precipitation toward the equator across the ensemble. 
We therefore hypothesize that a narrowing of the ITCZ may be an important contributor to increases in large-scale clustering of precipitation under warming. 
This implicates mechanisms related to the transport of energy by the Hadley circulation that have been argued to control changes in ITCZ width \citep{Byrne2016}.

On the other hand, the intermodel spread in changes in clustering with warming across the CMIP6 ensemble is related to zonal rather than meridional shifts in the precipitation. 
This motivated an investigation of the role played by Pacific SST gradients in changes in large-scale clustering of precipitation. 
In interannual variability, El Ni\~no-Southern Oscillation linked variability appears to be a major driver of variability in large-scale clustering of precipitation, with precipitation during El Ni\~no events more clustered than during La Ni\~na events. 

Under warming, changes in zonal SST gradients appeared to explain the sensitivity of projected clustering to zonal shifts in heavy rainfall; those models with more El Ni\~no-like warming patterns tended to exhibit stronger increases in precipitation clustering. 
This is important given the large disagreement between observed and simulated SST trends in the topical Pacific \citep[e.g.,][]{Wills2022}. 
Observations show a strengthening of the SST gradient, suggesting a weaker increase in large-scale clustering compared to simulations, which tend to show a weakening of tropical SST gradients.

Finally, we assessed if the changes in clustering with warming may have an influence on climate sensitivity. 
In observed interannual variability, a greater area fraction of heavy precipitation, $A_f$, is associated with a drier domain-mean and an increase in low-cloudiness in subsidence regions, LCF$_d$. 
The connection between clustering for a given $A_f$ persists for LCF$_d$, 
but changes in the mean area of precipitation features and meridional and zonal shifts in heavy precipitation generally have weak relationships to the tropical-mean relative humidity, RH, independent of their relationship to $A_f$. 

GCMs from the CMIP ensemble generally capture the observed tropical environment signatures associated with changes in $A_f$, but often have different RH and LCF$_d$ connections to shifts in heavy precipitation independent of $A_f$. 
In contrast to observations, RH in several models is sensitive to both meridional and zonal shifts in heavy precipitation. 
In CMIP6 models, zonal shifts of precipitation to the central Pacific tend to moisten, whereas meridional shifts to the equator tend to dry. 
Realistically represented or not, these sensitivities appear to affect how these models project relative humidity into the future; 
the subset of models sensitive to drying from meridional contraction of heavy precipitation create considerable spread in the model ensemble relative humidity changes under warming.

The study includes several limitations that are worth highlighting. 
Perhaps most importantly, the models we examined do not resolve the processes leading to organization of convection on mesoscales, 
which in turn may affect how they simulate heavy precipitation associated with large-scale convective features \citep{Bao2017}. 
This includes the high-resolution GCM, which still employs a parameterized convection scheme \citep{IFS_ref}. 
Another limitation is that monthly anomalies from the climatology of the associated month obscure variations in diurnal and daily clustering tendencies and seasonal differences in the strength of relationships. 
Similarly, the climatological values do not control for variations in the contribution from different timescales, including diurnal up to seasonal and decadal variations in clustering. 
Finally, we note that our model ensemble is one of opportunity, and the models used were dictated by the available data. 
Correlation across the ensemble is not guaranteed to be produced by a physical relationship, and the extent to which such relationships arise by chance rise the more variables are examined. 
Nevertheless, the relationships between SST gradients and shifts in the precipitation distribution we highlight here are based on well-established physical relationships that provides some confidence in their robustness.

Future research is encouraged to adopt the control for total convective area, or other similar controls for changes in the mean precipitation rate, as used in the present framework. 
One avenue for further investigation is to identify models with realistic clustering compared to observations. 
The CMIP6 models considered here show a wide range in climatological clustering and internal variability in clustering, and perhaps a subset of models with more realistic clustering characteristics should be given more weight in projections of climate. 
In a similar way, investigating the connection between large-scale clustering and mesoscale clustering in high-resolution observations and storm resolving models may further constrain the model spread in projections by identifying unrealistic behavior. 
Further developing these research endeavors would allow for increased confidence and reduce the model uncertainty in aspects of projections that could be influenced by changes in convective organization, 
ultimately allowing for improvement in mitigation and adaptation strategies for a warming climate.


